% ======================================================================
% ======================================================================
% ======================================================================
%
% UNOFFICIAL style for beamer presentations at RU-DI
% Radboud University, Donders Institute,
% Nijmengen, Netherlands
%
%     Author: Pierre Guetschel <pierre.guetschel@donders.ru.nl>
%       Date: Oct 2022
% Instructions:
% - to create the title page, use: \titleframe
% - to create a section page, use: \sectionframe
%
% Inspirations:
% - https://tex.stackexchange.com/questions/146529/design-a-custom-beamer-theme-from-scratch
% - 
%
% ======================================================================
% ======================================================================
% ======================================================================
\documentclass{beamer}
\usepackage[utf8]{inputenc}
\usepackage[T1]{fontenc}


\usetheme{rudi} % Radboud University, Donders Institute  beamer theme

%%% Uncomment to remove the foot line: %%%
%\setbeamertemplate{footline}{}

%%% Uncomment to remove the bands on the left: %%%
%\setbeamertemplate{sidebar canvas left}{}


\title[Beamer theme for D.I.]{A beamer theme for the Donders Institute!}
\author[Guetschel]{Pierre~Guetschel}
\institute[Donders Institute]
{ 
	Radboud University, Donders Institute, Nijmegen, Netherlands
}

\begin{document}

\titleframe % Use this command to create the title page

\section{Beamer elements} 
\sectionframe % Use this command to create a section page 

\begin{frame} 
	\frametitle{The beamer blocks} 
	\framesubtitle{This is the slide subtitle} 
	\begin{theorem}
		This is a standard block (here a theorem). 
	\end{theorem} 
	\begin{exampleblock}{Example}
		Here is an example block
	\end{exampleblock} 
	Here is a bit of text between the blocks.
	\begin{alertblock}{Alert}
		Here is an alert block.
	\end{alertblock} 
\end{frame}

\begin{frame}{Itemize and Enumerate}
	\begin{enumerate} 
	\item<1-> Suppose $p$ were the largest prime number. 
	\item<2-| alert@2> Let $q$ be the product of the first $p$ numbers. 
	\item<3-> Then $q+1$ is not divisible by any of them. 
	\begin{itemize}
	\item 3-a
	\begin{enumerate}
	\item here
	\item there
	\end{enumerate}
	\item 3-b
	\end{itemize}
	\item<1-> But $q + 1$ is greater than $1$, thus divisible by some prime
	number not in the first $p$ numbers.
	\end{enumerate}
\end{frame}

\begin{frame}{The figures and tables}
	\begin{figure}
		\includegraphics[width=4cm]{example-image-a}
		\caption{Example figure's caption}
	\end{figure}

\end{frame}


\section{Theme colors} 
\sectionframe

\begin{frame}{Available colors}
Base colors:\\
\colorbox{rudi-black}{\color{white} rudi-black}
\colorbox{rudi-dark-red}{\color{white} rudi-dark-red}
\colorbox{rudi-ru-red}{\color{white} rudi-ru-red}
\colorbox{rudi-bright-red}{\color{white} rudi-bright-red}
\vspace{1em}

Extended colors:\\
\colorbox{rudi-dark-orange}{\color{white} rudi-dark-orange}
\colorbox{rudi-light-orange}{rudi-light-orange}
\colorbox{rudi-yellow}{rudi-yellow}
\colorbox{rudi-blue}{\color{white} rudi-blue}
\colorbox{rudi-green}{\color{white} rudi-green}
\vspace{1em}

Watered colours:\\
\colorbox{rudi-water-red}{rudi-water-red}
\colorbox{rudi-water-orange}{rudi-water-orange}
\colorbox{rudi-water-yellow}{rudi-water-yellow}
\colorbox{rudi-water-blue}{rudi-water-blue}
\colorbox{rudi-water-green}{rudi-water-green}
\end{frame}

\end{document}