% ======================================================================
% ======================================================================
% ======================================================================
%
% UNOFFICIAL style for beamer presentations at RU-DI
% Radboud University, Donders Institute,
% Nijmengen, Netherlands
%
%     Author: Pierre Guetschel <pierre.guetschel@donders.ru.nl>
%       Date: Oct 2022
% Instructions:
% - to create the title page, use: \titleframe
% - to create a section page, use: \sectionframe
%
% Inspirations:
% - https://tex.stackexchange.com/questions/146529/design-a-custom-beamer-theme-from-scratch
% - 
%
% ======================================================================
% ======================================================================
% ======================================================================
\documentclass{beamer}
\usepackage[utf8]{inputenc}
\usepackage[T1]{fontenc}


\usetheme{rudi} % Radboud University, Donders Institute  beamer theme

%%% Uncomment to remove the foot line: %%%
%\setbeamertemplate{footline}{}

%%% Uncomment to remove the bands on the left: %%%
%\setbeamertemplate{sidebar canvas left}{}


\title[Beamer theme for D.I.]{A beamer theme for the Donders Institute!}
\author[Guetschel]{Pierre~Guetschel}
\institute[Donders Institute]
{ 
	Radboud University, Donders Institute, Nijmegen, Netherlands
}

\begin{document}

\titleframe % Use this command to create the title page

\section{The first section} 
\sectionframe % Use this command to create a section page 

\begin{frame} 
	\frametitle{There Is No Largest Prime Number} 
	\framesubtitle{The proof uses \textit{reductio ad absurdum}.} 
	\begin{theorem}
	There is no largest prime number. 
	\end{theorem} 
	\begin{enumerate} 
	\item<1-| alert@1> Suppose $p$ were the largest prime number. 
	\item<2-> Let $q$ be the product of the first $p$ numbers. 
	\item<3-> Then $q+1$ is not divisible by any of them. 
	\begin{itemize}
	\item 3-a
	\begin{enumerate}
	\item here
	\item there
	\end{enumerate}
	\item 3-b
	\end{itemize}
	\item<1-> But $q + 1$ is greater than $1$, thus divisible by some prime
	number not in the first $p$ numbers.
	\end{enumerate}
\end{frame}

\section{The second section} 
\sectionframe

\begin{frame}{The last frame's title}
	\begin{figure}
		\includegraphics[width=4cm]{example-image-a}
		\caption{Example figure's caption}
	\end{figure}
	\begin{itemize}
		\item one
		\item two
		\begin{itemize}
			\item two-a
			\begin{itemize}
				\item two-a-1
				\item two-b-2
			\end{itemize}
			\item two-b
		\end{itemize}
	\end{itemize}
\end{frame}

\end{document}